\documentclass[11pt]{article}
\usepackage{ragged2e}
\usepackage[utf8]{inputenc}
\usepackage[catalan]{babel}
\usepackage{hyperref}
\usepackage{listings}
\usepackage{afterpage}
\usepackage{pgfplots}
\pgfdeclarelayer{background}
\pgfdeclarelayer{foreground}
\pgfsetlayers{background,main,foreground}
\begin{document}
	\begin{titlepage}
		\newcommand{\HRule}{\rule{\linewidth}{0.5mm}} % Defines a new command for the horizontal lines, change thickness here
		
		\center % Center everything on the page
		
		%----------------------------------------------------------------------------------------
		%	HEADING SECTIONS
		%----------------------------------------------------------------------------------------
		
		\textsc{\LARGE Universitat de Lleida}\\[1.5cm] % Name of your university/college
		\includegraphics{Images/LogoUDL.jpg}\\[1cm] % Include a department/university logo - this will require the graphicx package
		\textsc{\Large Grau en Enginyeria Informàtica}\\[0.5cm] % Major heading such as course name
		\textsc{\large Aprenentatge i Raonament Automàtic}\\[0.5cm] % Minor heading such as course title
		
		%----------------------------------------------------------------------------------------
		%	TITLE SECTION
		%----------------------------------------------------------------------------------------
		
		\HRule \\[0.4cm]
		{ \huge \bfseries Pràctica 2, Festa major a Hivèrnia}\\[0.4cm] % Title of your document
		\HRule \\[1.5cm] 
		%----------------------------------------------------------------------------------------
		%	AUTHOR SECTION
		%----------------------------------------------------------------------------------------
		
		\begin{minipage}{0.4\textwidth}
			\begin{flushleft} \large
				\emph{Autors:}\\
				Jordi Ricard Onrubia Palacios\\
				Marcel Porta Valles
			\end{flushleft}
		\end{minipage}
		~
		\begin{minipage}{0.4\textwidth}
			\begin{flushright} \large
				\emph{Professor:} \\
				Ramón Bejar Torres
			\end{flushright}
		\end{minipage}\\[4cm]
		
		%----------------------------------------------------------------------------------------
		%	DATE SECTION
		%----------------------------------------------------------------------------------------
		{\large \today}\\[3cm] % Date, change the \today to a set date if you want to be precise
		\vfill % Fill the rest of the page with whitespace
	\end{titlepage}
\newpage
\thispagestyle{empty}
\newpage
\tableofcontents
\listoffigures
\newpage
\clearpage
\pagenumbering{arabic}
\section{Funcions:}
\subsection{execSeqofSteps:}
Funció recursiva que cridarà updateSequenceOfSteps per tal d'executar tots els passos, aquesta finalitzarà quan la llista de passos es quedi buida.\\
\textbf{Arguments:}
\begin{itemize}
\item PrevLocs: Matriu NxN que representa el món en el qual em de trobar a Bàrcenas, aquesta inicialment serà una matriu tot a uns a excepció de la posició 1,1 que serà 0 sent aquesta la posició inicial.
\item Llista de passos: Llista de passos a seguir per tal de buscar a Bàrcenas, aquesta es compon de X, Y direccions a les quals ens anem desplaçant per a realitzar la cerca, S, detector d'olor de Bàrcenas, M i C, respostes donades per Mariano i Cospedal.
\item FinalLocs: Matriu NxN actualitzada amb les noves possibles localitzacions de Bàrcenas després d'executar cada un dels passos de la llista.
\item Pas[]: Aquests arguments ens permeten guardar la Y on es va trobar Rajoy, la seva resposta així com la resposta de Cospedal.
\end{itemize}
\subsection{updateSequenceOfSteps:}
Funció que s'encarrega de processar el pas passat per argument, aquest executarà totes les altres funcions per tal de realitzar aquest procés.\\
\textbf{Arguments:}
\begin{itemize}
\item PrevLocs: Matriu NxN que representa el món en el qual em de trobar a Bàrcenas, aquesta inicialment serà una matriu tot a uns a excepció de la posició 1,1 que serà 0 sent aquesta la posició inicial.
\item SequenceOfSteps: Passos a realitzar en aquell moment, aquests es componen de les posicions X, Y del món on hem de buscar a Bàrcenas, S detecto d'olor, M Mariano i C Cospedal.
\item Pas[]: Argument on es passen els estats previs per a la columna on hem trobat a Mariano, la resposta de Mariano i la resposta de Cospedal.
\item Fut[]: Argument on es guardarà la Columna on hem trobat a Mariano en cas que el trobem, la resposta de Mariano si l'hem trobat, en cas contrari guardarem -1, i la resposta de Cospedal si l'hem trobat, en cas contrari guardem -1.
\item FinalLocs: Estat del món un cop realitzat el pas passat per argument amb les possibles localitzacions de Bàrcenas.
\end{itemize}
\subsection{isBarcenasAround:}
\subsection{intersectLocs:}
\subsection{rajoyInfo:}
\subsection{cospedalInfo:}
\subsection{rajoyAndCospedal:}
\subsection{intersectLocs:}
\subsection{writeWorld:}
\end{document}